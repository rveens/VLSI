\documentclass[a4paper,twoside,11pt]{article}
\usepackage{a4wide,graphicx,fancyhdr,amsmath,amssymb}

%----------------------- Macros and Definitions --------------------------

\setlength\headheight{20pt}
\addtolength\topmargin{-10pt}
\addtolength\footskip{20pt}

\newcommand{\N}{\mathbb{N}}
\newcommand{\ch}{\mathcal{CH}}

\newcommand{\solution}[1]{\noindent{\bf Solution to Exercise #1:}}

\fancypagestyle{plain}{%
\fancyhf{}
\fancyhead[LO,RE]{\sffamily\bfseries\large technische universiteit eindhoven}
\fancyhead[RO,LE]{\sffamily\bfseries\large 2IN35 VLSI}
\fancyfoot[LO,RE]{\sffamily\bfseries\large department of mathematics and computer science}
\fancyfoot[RO,LE]{\sffamily\bfseries\thepage}
\renewcommand{\headrulewidth}{0pt}
\renewcommand{\footrulewidth}{0pt}
}

\pagestyle{fancy}
\fancyhf{}
\fancyhead[RO,LE]{\sffamily\bfseries\large technische universiteit eindhoven}
\fancyhead[LO,RE]{\sffamily\bfseries\large 2IN35 VLSI}
\fancyfoot[LO,RE]{\sffamily\bfseries\large department of mathematics and computer science}
\fancyfoot[RO,LE]{\sffamily\bfseries\thepage}
\renewcommand{\headrulewidth}{1pt}
\renewcommand{\footrulewidth}{0pt}

\def\addsquare#1{\tikz\node[draw]{#1};} 

%-------------------------------- Title ----------------------------------

\title{\vspace{-\baselineskip}\sffamily\bfseries Assignment 3}
\author{
	Rick Veens \qquad Studentno: 0912292\\
	\texttt{r.veens@student.tue.nl}
	\and
	Barry de Bruin \qquad Studentno: -\\
	\texttt{-@student.tue.nl}
}

\date{\today}

%--------------------------------- Text ----------------------------------

\begin{document}
\maketitle
\newpage

\section*{Lab assignment 3a}
\subsection*{1. Requirements}
The assignment is to design and implement a FIR filter named filter that:
\begin{enumerate}
\item Uses as little resources as possible and is maximally sequential. In particular at most 1
multiplier may be used.
\item Conforms to the 4-phase asynchronous protocol for both input and output.
\item Can run at a clock frequency of 100 Mhz.
\item Honors changes in the coefficients (after a finite delay).
\item May produce a finite length interval of start-up noise
\end{enumerate}

\subsubsection*{1.1 Analysis of requirements}


\subsection*{3. System architecture}
Block diagram
\subsection*{4. Design choices}

\subsection*{5. Functional correctness}
all requirements?

\subsection*{6. Resource usage}


\subsection*{7. System throughput and latency}
 (min) sample time Ts and (max) sample
frequency fs, both after synthesis and after placement \& routing.

\subsection*{8. Simulation results}
Time en FFT plot


\newpage
\section*{Lab assignment 3b}
\subsection*{1. Requirements}
The specific requirements for the strength reduced FIR filter are:
\begin{enumerate}
\item The design may use at most 3 multipliers.
\item Conforms to the 4-phase asynchronous handshake protocol for both input and output.
\item Can run at a clock frequency of 100 Mhz.
\item Should have about 4 times the sample frequency as the sequential implementation.
\item Honors changes in the coefficients (after a finite delay).
\item May produce a finite length interval of start-up noise.
\end{enumerate}


\newpage
\section*{Comparison between two filters}
In your final report, compare the output signals of the strength reduced filter with the output of the sequential filter, by generating a difference signal for a representative subset of
samples. Make sure you align the outputs correctly, because the amount of start-up noise of
the 2 implementations may differ. If the difference signal is non-zero, explain the differences.
Further reporting guidelines are on the course’s website.

\newpage
\section*{Appendix}
all verilog code



\end{document}
